\documentclass{article}
\usepackage{enumerate}
\usepackage{amsmath, amsthm, amssymb}
\usepackage[margin=1in]{geometry}
\usepackage[parfill]{parskip}

\title{Econ C103 Problem Set 2}
\author{Sahil Chinoy}

\begin{document}
\maketitle{}

\subsection*{Exercise 1}

\begin{enumerate}[(a)]

	\item

	In this context, a mechanism consists of

	\begin{itemize}

		\item A set of messages $M$.

		\item A rule that maps from the set of messages to the (real-valued) tax the worker pays $t: M \rightarrow \mathbb{R}$.

		\item A rule that maps from the set of messages to the type of job the worker has $x: M \rightarrow \{l,h\}$.

	\end{itemize}

	\item

	A direct mechanism is one in which the worker reports their ability $\theta$. This implies $M = [\underline{\theta}, \bar{\theta}]$.

	\item

	An incentive compatible direct mechanism is one in which it is always optimal for the worker to tell the truth about their ability, that is, $x(m) = x(\theta)$. This implies

	\begin{equation*}
	w_{x(\theta)} - t(\theta) - \frac{x(\theta)}{
	\theta} \geq \max_m \left \{ w_{x(m)} - t(m) - \frac{x(m)}{\theta} \right \}
	\end{equation*}

	\item

	Incentive compatibility implies that an individual with true ability $\theta$ should prefer to report $\theta \neq \theta'$.

	\begin{equation*}
	w_{x(\theta)} - t(\theta) - \frac{x(\theta)}{
	\theta} \geq w_{x(\theta')} - t(\theta') - \frac{x(\theta')}{
	\theta}
	\end{equation*}
	\begin{equation}
	\frac{1}{\theta} (x(\theta') - x(\theta)) \geq w_{x(\theta')} - t(\theta') - (w_{x(\theta)} - t(\theta))
	\end{equation}

	Likewise, an individual with true ability $\theta'$ should prefer to report $\theta' \neq \theta$.

	\begin{equation*}
	w_{x(\theta')} - t(\theta') - \frac{x(\theta')}{
	\theta'} \geq w_{x(\theta)} - t(\theta) - \frac{x(\theta)}{
	\theta'}
	\end{equation*}
	\begin{equation}
	\frac{1}{\theta'} (x(\theta) - x(\theta')) \geq w_{x(\theta)} - t(\theta) - (w_{x(\theta')} - t(\theta'))
	\end{equation}

	Adding (1) and (2)

	\begin{equation*}
	(\frac{1}{\theta} - \frac{1}{\theta'})(x(\theta') - x(\theta)) \geq 0
	\end{equation*}

	Since $0 \leq \underline{\theta}$, both $\theta$ and $\theta'$ are positive, so we can rearrange to find

	\begin{equation*}
	(\theta'- \theta)(x(\theta') - x(\theta)) \geq 0
	\end{equation*}

	Thus, for $\theta' > \theta$, $x(\theta') > x(\theta)$. This proves $x(\theta)$ is monotonically increasing.

	\item 

	Since $x = \{l,h\}$, we define $\theta_0 = \sup \{ \theta: x(\theta) = l \}$. Since a worker will always report the type that leads to the lowest taxes, the tax can only depend on the type of job the worker has, not on their ability.

	\begin{equation*}
	t(\theta) =  \begin{cases} 
	      t_1, &\theta > \theta_0 \\
	      t_0, & \theta < \theta_0
	   \end{cases}
	\end{equation*}

	Workers with ability $\theta < \theta_0$ should prefer the low-intensity job and associated tax

	\begin{equation}
		w_l - t_0 - \frac{l}{\theta_0} \geq w_h - t_1 - \frac{h}{\theta_0}
	\end{equation}

	Likewise, workers with ability $\theta > \theta_0$ should prefer the high-intensity job and associated tax

	\begin{equation}
		w_l - t_0 - \frac{l}{\theta_0} \leq w_h - t_1 - \frac{h}{\theta_0}
	\end{equation}

	Combining (3) and (4)

	\begin{equation*}
		w_l - t_0 - \frac{l}{\theta_0} = w_h - t_1 - \frac{h}{\theta_0}
	\end{equation*}
	\begin{equation*}
		t_1 = t_0 + (w_h - w_l) - \frac{h-l}{\theta_0}
	\end{equation*}

	So

	\begin{equation*}
	t(\theta) =  \begin{cases} 
	      t_0 + (w_h - w_l) - \frac{h-l}{\theta_0}, &\theta > \theta_0 \\
	      t_0, & \theta < \theta_0
	   \end{cases}
	\end{equation*}

	\item

	Since the outcome of any mechanism can be implemented in an incentive-compatible direct mechanism, the set of outcomes that can be implemented in any mechanism is characterized by the job allocation rule

	\begin{equation*}
	x(\theta) =  \begin{cases} 
	      h, &\theta > \theta_0 \\
	      l, & \theta < \theta_0
	   \end{cases}
	\end{equation*}

	And the taxation rule

	\begin{equation*}
	t(\theta) =  \begin{cases} 
	      t_0 + (w_h - w_l) - \frac{h-l}{\theta_0}, &\theta > \theta_0 \\
	      t_0, & \theta < \theta_0
	   \end{cases}
	\end{equation*}

	\item

	Consider an indirect mechanism where the worker decides between high and low intensity jobs and reports only their wage $w_x \in \mathbb{R}$. Then $M = \mathbb{R}$. One outcome that can be implemented in this mechanism is a uniform tax $\tau$ on wages

	\begin{equation*}
	t(w_x) = \tau w_x
	\end{equation*}

	The allocation rule is meaningless, since the worker decides their own job.

	\item

	The participation constraint implies the agent has positive utility in all cases

	\begin{equation*}
	\max_m \left \{ w_{x(m)} - t(m) - \frac{x(m)}{\theta} \right \} \geq 0
	\end{equation*}

	In the context of taxation, this means that wages net of taxes and the cost of effort from working must be positive. If we consider a world in which unemployment is an option, i.e. the worker can choose to hold no job, receive no wage, and pay no tax ($x = w_x = t =0$), then the participation constraint is binding in the sense that an agent can opt for unemployment and thereby refuse to participate in the mechanism. If we require that everyone hold a job and pay taxes, then the participation constraint does not apply.

	\item

	For $l = w_l = 0$, we have

	\begin{equation*}
	E[t] = F(\theta_0) t_0 + (1 - F(\theta_0)) (t_0 + w_h - \frac{h}{\theta_0})
	\end{equation*}

	If we restrict ourselves to mechanisms where $t \leq w_{x(\theta)}$, then $t_0 \leq 0$, since $w_l = 0$. Clearly, the expected tax revenue is increasing in $t_0$, so the maximum expected revenue occurs for $t_0 = 0$. This implies a tax structure

	\begin{equation*}
	t(\theta) =  \begin{cases} 
	      w_h - \frac{h}{\theta_0}, &\theta > \theta_0 \\
	      0, & \theta < \theta_0
	   \end{cases}
	\end{equation*}

	\item

	When $t=0$, the worker will choose the high intensity job if

	\begin{equation*}
	w_h - \frac{h}{\theta} > 0
	\end{equation*}
	\begin{equation*}
	\theta > \frac{h}{w_h}
	\end{equation*}

	With $w_h = 4$ and $h = 1$, workers with ability $\theta > \frac{1}{4}$ will choose the high intensity job. Given that $\theta$ is uniformly distributed, this implies the worker will choose the high intensity job with probability $\frac{3}{4}$.

	\item 

	With $t_0 = 0$, $w_h = 4$, $h=1$, $F(\theta) = \theta$

	\begin{equation*}
	E[t] = (1 - \theta_0) (4 - \frac{1}{\theta_0}) = 5 - 4\theta_0 - \frac{1}{\theta_0}
	\end{equation*}

	The expected revenue is maximized for $\frac{d E[t]}{d \theta_0} = 0$

	\begin{equation*}
	-4 + \frac{1}{\theta_0^2} = 0
	\end{equation*}
	\begin{equation*}
	\theta_0 = \frac{1}{2}
	\end{equation*}

	So, workers with ability $\theta > \frac{1}{2}$ will choose the high intensity job. Given that $\theta$ is uniformly distributed, this implies the worker will choose the high intensity job with probability $\frac{1}{2}$.

\end{enumerate}

\end{document}